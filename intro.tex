\section{Introduction}
Energy efficiency in the buildings sector offers great potential for cost-effective emissions reductions. In buildings we spend ~90\% of our time, consume ~75\% of total electricity, which represents nearly half of our primary energy consumption, and generate 45\% of our CO2 emissions ~\cite{efficiency2009buildings}. In large commercial buildings, traditional digital control systems regulate the majority of the energy use, particularly HVAC systems, which we focus on here, and lighting. These large sensor deployments are cyber-physical systems with thousands nodes. 
Software applications have been recently developed to optimize energy use, improve comfort, and identify faults for these systems ~\cite{krioukov2012building,dawson2013boss,weng2013buildingdepot,arjunan2012sensoract,wheeler1992understanding}. For all these applications to be implemented detailed information about sensor context is required. Unfortunately, such information is very difficult to obtain, because it is either embedded in the building automation system (BAS) or lacking. Current efforts in automatic metadata acquisition include two different strategies: extrapolating metadata from labels (e.g. BACnet point names), and inferring them from sensor readings. Recent work has just started exploring the latter approach. Fontugne et al. ~\cite{EMD} proposed a method to correlate inter-device user patterns by extracting traces of occupancy from electrical energy use. Koc et al. ~\cite{koc2014comparison} compared correlation methods to infer spatial relationships between discharge and zone temperature sensors in different rooms of a building. Rajagopal et al. ~\cite{rajagopal2014visual} developed a method for using LED frequency modulation and smartphones cameras to establish a relationship between fixture and occupant location.
Despite these efforts, many issues remain unresolved. In particular, we set out to devise a method for inferring functional relationships between HVAC components, as lack of this information precludes the adoption of common energy efficiency strategies (resets). This paper explores an instance of this problem (on purpose this is a very difficult case to solve). The example is specific to a particular building, but technique and considerations are generalizable to many other buildings. Indeed, the majority of the US large commercial buildings have some variation of these systems installed.
