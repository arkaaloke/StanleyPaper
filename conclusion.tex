\section{Conclusion and Future Work}

In this paper we present a novel algorithm to infer relationships between HVAC components of large commercial buildings. We show that due to the characteristics of the data (response lags, nested control loops, tight variable boundaries) other common techniques are not effective in this context. The new algorithm utilizes perturbations of AHU variables and guarantees that the building zones remain within comfort. The method was applied to an existing building and its results, being able to identify the relationship correctly in ~80\% of the cases.

There are three main directions of research we would like to explore in the future. First, we would like to explore how our technique generalizes to other buildings, where zones may have different sensors, setpoints and control loops. Second, we would like to explore whether we can discover relationships between subsystems like the air handling units and variable air volume units without having to introduce a large perturbation to the air handling unit. We may be able to utilize milder perturbations like daily occupant cycles, and techniques from statistical process control to infer the same relationships. Finally, a useful future direction of research is to quantify the amount of cross-talk between zones. The heat exchange between zones with common doors, passageways, etc introduce errors into any single-zone heat exchange analysis, and its identification and quantification could lead to more efficient and better designed control loops. 
